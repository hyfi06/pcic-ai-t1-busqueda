\documentclass[spanish]{article}
\usepackage[T1]{fontenc}
\usepackage[utf8]{inputenc}
\usepackage[spanish,es-tilden,es-lcroman]{babel}
\usepackage[letterpaper,top=3.5cm,bottom=3.5cm,marginpar=3cm,left=3.5cm,right=3.5cm]{geometry}
\usepackage{amsmath,amssymb,units,amsfonts}
\usepackage{wasysym}
\title{Ejercicios de Lógica}
\author{Héctor Olvera Vital}
\begin{document}

\maketitle


\section*{7.7}
Como $A$, $B$, $C$, $D$ son los únicos proposiciones, existen $2^4$ modelos.

\subsection*{a) $B \vee C$}

\begin{tabular}{c|c|c}
  B & C & $B \vee C$ \\\hline
  V & V & V          \\
  V & F & V          \\
  F & V & V          \\
  F & F & F
\end{tabular}

Hay 3 modelos que satisface la fórmula. Como $A$ y $D$ no aparecen en la formula, entonces en total hay $3*2^2=12$ modelos en total.

\subsection*{ b) $\neg A \vee \neg B \vee \neg C \vee \neg D$}

La fórmula es equivalente a $\neg (A \wedge B \wedge C \wedge D)$.

Si un modelo no satisface la fórmula, entonces satisface $A\wedge B\wedge C\wedge D$, entonces el modelo es $s(A)=T$, $s(B)=T$, $s(C)=T$ y $s(D)=T$

Por lo que, todos los demás modelos satisfacen la fórmula. En total $2^4 -1 = 15$

\subsection*{ c) $(A \rightarrow B)\wedge A \wedge \neg B \wedge C \wedge D$}

Supongamos que $s$ una asignación de verdad que satisface la fórmula.

Entonces,

\begin{align*}
  s(A \rightarrow B) = T \\
  s(A) = T               \\
  s(\neg B) = T          \\
  s(C) = T               \\
  s(D) = D
\end{align*}

Pero $s(A) = T$ y $s(\neg B) = T$, por lo que $s(A\rightarrow B) = F$. Lo cual es una contradicción.

Por lo tanto no existe la asignación.

Entonces existen 0 modelos para la fórmula.

\section*{7.4}

\subsection*{ a) $False \models True$}

Verdadero.

$\{False, \neg True\}$ no es satisfactible.

\subsection*{ b) $True \models False$}

Falso. $\{ True, \neg False\}$ es satisfactible. Sea $w=\{\}$, $I(True\wedge \neg False, w)=1$

\subsection*{c) $A\wedge B \models A \leftrightarrow B$}

Verdadero.

Sea w un modelo. Si $I(A\wedge B,w)=1$, entonces $I(A,w) = 1$ y $I(B,w)=1$. Por lo tanto, $I(A\leftrightarrow B,w) = 1$.

\subsection*{ d) $A\leftrightarrow B \models A \vee B$}

Falso. $\{A\leftrightarrow B, \neg (A\vee B)\}$ es satisfactible.

Sae $w = \{A:F, B:F\}$. Entonces $I(A\leftrightarrow B,w)= 1$ y $I(A\vee B,w)=0$. Por lo tanto, $I(\neg (A\vee B),w)=1$ y el conjunto es satisfactible.

\subsection*{ e) $A\leftrightarrow B \models \neg A \vee B$}

Verdadero.

Veamos que $\{A\leftrightarrow B, \neg (\neg A \vee B)\}$ no es satisfactible.

Si $I(\neg (\neg A \vee B),w) =1$, entonces $I(A\wedge \neg B,w) =1$. Por lo que, $I(A,w) =1$ y $I(B,w)=0$. Entonces $I(A\leftrightarrow B,w)=0$.

Por lo que, el conjunto no es satisfactible.

\subsection*{ f) $(A \wedge B) \rightarrow C \models (A \rightarrow C) \vee (B \rightarrow C)$ }

Verdadero.

Veamos que $\{(A \wedge B) \rightarrow C, \neg ((A \rightarrow C) \vee (B \rightarrow C))\}$ no es satisfactible.

Sea $w$ un modelo. Si $I(\neg ((A \rightarrow C) \vee (B \rightarrow C)),w) = 1$, entonces $I(\neg( A \rightarrow C),w) =1$ y $I(\neg (B \rightarrow C),w)=1$.

Entonces, $I(A\wedge \neg C,w) = 1$ y $I(B\wedge \neg C,w)=1$. Por lo tanto, $I(A,w)=1$, $I(B,w)=1$ y $I(C,w)=0$.

Entonces $I((A\wedge B)\rightarrow C)=0$. Por lo tanto el conjunto no es satisfactible.

\subsection*{ g) $C \vee (\neg A \wedge \neg B) \equiv A \rightarrow C\wedge B\rightarrow C$}

Sae $w$ tal que $I(C \vee (\neg A \wedge \neg B),w)=1$. Entonces, $I(C,w)=1$ o $I(\neg A \wedge \neg B,w)=1$

Si $I(C,w) = 1$, $I(A\rightarrow C,w) = 1$ y $I(B\rightarrow C,w)=1$. Por lo que, $I(A \rightarrow C\wedge B\rightarrow C, w)=1$.

Si $I(\neg A \wedge \neg B, w)=1$, entonces $I(A)=0$ y $I(B)=0$. Por lo que, $I(A\rightarrow C,w) =1 $ y $I(B\rightarrow C,w)=1$. Así, $I(A \rightarrow C\wedge B\rightarrow C, w)=1$.

En ambos casos, $I(A \rightarrow C\wedge B\rightarrow C, w)=1$.

Por lo que, $C \vee (\neg A \wedge \neg B) \models A \rightarrow C\wedge B\rightarrow C$

Sea $w$ tal que $I(A \rightarrow C\wedge B\rightarrow C, w)=1$. Entonces $I(A\rightarrow C,w) = 1$ y $I(B\rightarrow C,w)=1$. Por lo que, $I(\neg A\vee C,w) = 1$ y $I(\neg B\vee C,w)=1$.

Si $I(C,w)=1$, entonces $I(C \vee (\neg A \wedge \neg B),w)=1$.

Si $I(C,w)=0$, entonces $I(\neg A,w)=1$ y $I(\neg B,w)=1$. Por lo que, $I(C \vee (\neg A \wedge \neg B),w)=1$.

En ambos casos $I(C \vee (\neg A \wedge \neg B),w)=1$

Entonces, $A \rightarrow C\wedge B\rightarrow C\models C \vee (\neg A \wedge \neg B)$

Por lo tanto, $C \vee (\neg A \wedge \neg B) \equiv A \rightarrow C\wedge B\rightarrow C$

\subsection*{ h) $(A\vee B)\wedge (\neg C \vee \neg D \vee E)\models A \vee B$}

Verdadero.

Veamos que $\{(A\vee B)\wedge (\neg C \vee \neg D \vee E),\neg(A \vee B)\}$ no es satisfactible.

Sea $w$ un modelo. Si $I((A\vee B)\wedge (\neg C \vee \neg D \vee E),w)=1$, $I(A\vee B,w)=1$. Por lo que, $I(\neg(A\vee B),w)=0$.

Entonces el conjunto no es satisfactible.

\subsection*{ i) $(A\vee B)\wedge (\neg C \vee \neg D \vee E)\models (A \vee B)\wedge (\neg D \vee E)$}

Falso.

Sae $w=\{A:V,B:V,C:F,D:V,E:F\}$.

\begin{tabular}{ccccccccccc}
  $(A$ & $\vee$ & $B)$ & $\wedge$ & $(\neg$ & $C$ & $\vee$ & $\neg$ & $D$ & $\vee$ & $E)$ \\\hline
  V    & V      & V    & \bf V    & V       & F   & V      & F      & V   & F      & F
\end{tabular}


\begin{tabular}{cccccccc}
  $(A$ & $\vee$ & $B)$ & $\wedge$ & $(\neg$ & $D$ & $\vee$ & $E)$ \\\hline
  V    & V      & V    & \bf F    & F       & V   & F      & F
\end{tabular}


\subsection*{ j) $(A\vee B) \wedge \neg (A\rightarrow B)$}

Verdadero.

Sae $w=\{A:V,B:F\}$.

\begin{tabular}{cccccccc}
  $(A$ & $\vee$ & $B)$ & $\wedge$ & $\neg$ & $(A$ & $\rightarrow$ & $B)$ \\\hline
  V    & V      & F    & \bf V    & V      & V    & F             & F
\end{tabular}



\subsection*{ k) $(A\leftrightarrow B )\wedge (\neg A\vee B)$}
Verdadero.


Sae $w=\{A:V, B:V\}$.

\begin{tabular}{cccccccc}
  $(A$ & $\leftrightarrow$ & $B)$ & $\wedge$ & $(\neg$ & $A$ & $\vee$ & $B)$ \\\hline
  V    & V                 & V    & \bf V    & F       & V   & V      & V    \\
\end{tabular}

\subsection*{ l) }

Verdadero.

Por inducción sobre el número de símbolos proposicionales adicionales a $A$, $B$, $C$.

Paso base $n=0$

\begin{tabular}{c|c|c|c|c}
  $A$ & $B$ & $C$ & $(A\leftrightarrow B)$ & $(A\leftrightarrow B)\leftrightarrow C$ \\\hline
  V   & V   & V   & V                      & V                                       \\
  V   & V   & F   & V                      & F                                       \\
  V   & F   & V   & F                      & F                                       \\
  V   & F   & F   & F                      & V                                       \\
  F   & V   & V   & F                      & F                                       \\
  F   & V   & F   & F                      & V                                       \\
  F   & F   & V   & V                      & V                                       \\
  F   & F   & F   & V                      & F                                       \\
\end{tabular}


Existen 4 modelos que satisface $(A\leftrightarrow B)$ y 4 modelos que satisface $(A\leftrightarrow B)\leftrightarrow C$.

H.I.: Supongamos para $n=k$ que tiene la misma cantidad de modelos.

Sean $A_1$, $A_2$,..., $A_k$, $A_{k+1}$, los símbolos proposicionales adicionales a $A$, $B$, $C$.
Por hipótesis de inducción, existen la misma cantidad de modelos para $(A\leftrightarrow B)$ y $(A\leftrightarrow B)\leftrightarrow C$ con $A$, $B$, $C$, $A_1$, $A_2$,..., $A_k$.
Al agregar $A_{k+1}$, se multiplica por 2 la cantidad de modelos posibles. Ya que como $A_{k+1}$ no aparece en $(A\leftrightarrow B)$ ni $(A\leftrightarrow B)\leftrightarrow C$, si w es un modelo que satisface alguna de las dos fórmulas, $w\cup\{A_{k+1}:True\}$ y $w\cup\{A_{k+1}:Falso\}$ son también modelos.

Entonces, existen la misma cantidad de model para $(A\leftrightarrow B)$ y $(A\leftrightarrow B)\leftrightarrow C$ con $A$, $B$, $C$, $A_1$, $A_2$,..., $A_k$, $A_{k+1}$.

Por el principio de inducción, el se cumple para toda $n$.

\section*{7.18}

$[(Food \rightarrow Party)\vee(Drinks\rightarrow Party)] \rightarrow [(Food \wedge Drinks)\rightarrow Party]$

\subsection*{ a)}
\begin{tabular}{ccccccccccccc}
  $[(Food$ & $\rightarrow$ & $Party)$ & $\vee$ & $(Drinks$ & $\rightarrow$ & $Party)]$ & $\rightarrow$ & $[(Food$ & $\wedge$ & $Drinks)$ & $\rightarrow$ & $Party]$ \\\hline
  V        & V             & V        & V      & V         & V             &           & V             &          & V        &           & V                        \\
  V        & V             & V        & V      & F         & V             &           & V             &          & F        &           & V                        \\
  V        & F             & F        & F      & V         & F             &           & V             &          & V        &           & F                        \\
  V        & F             & F        & V      & F         & V             &           & V             &          & F        &           & V                        \\
  F        & V             & V        & V      & V         & V             &           & V             &          & F        &           & V                        \\
  F        & V             & V        & V      & F         & V             &           & V             &          & F        &           & V                        \\
  F        & V             & F        & V      & V         & F             &           & V             &          & F        &           & V                        \\
  F        & V             & F        & V      & F         & V             &           & V             &          & F        &           & V                        \\
\end{tabular}

Es válida.

\subsection*{ b)}

$[(Food \rightarrow Party)\vee(Drinks\rightarrow Party)] \rightarrow [(Food \wedge Drinks)\rightarrow Party]$

$[(\neg Food \vee Party)\vee(\neg Drinks\vee Party)] \rightarrow [\neg (Food \wedge Drinks)\vee Party]$

$(\neg Food \vee Party\vee\neg Drinks\vee Party) \rightarrow (\neg Food \vee \neg Drinks \vee Party)$

Se confirma la a) ya que el consecuente de la implicación es el antecedente quitando la redundancia de Party.

\subsection*{ c)}

$\vdash (\neg Food \vee Party\vee\neg Drinks\vee Party) \rightarrow (\neg Food \vee \neg Drinks \vee Party)$

$\{\neg Food \vee Party\vee\neg Drinks\vee Party\} \vdash \neg Food \vee \neg Drinks \vee Party$

Apliquemos resolución

$\{\neg Food \vee Party\vee\neg Drinks\vee Party, \neg(\neg Food \vee \neg Drinks \vee Party)\}$

$\{\neg Food \vee Party\vee\neg Drinks\vee Party,  Food \wedge  Drinks \wedge \neg Party\}$

$\{\neg Food \vee Party\vee\neg Drinks\vee Party,  Food,  Drinks, \neg Party\}$

$Food$, $\neg Food \vee Party\vee\neg Drinks\vee Party$ -> $Party\vee\neg Drinks\vee Party$

$\neg Party$, $Party\vee\neg Drinks\vee Party$ -> $\neg Drinks\vee Party$

Drinks, $\neg Drinks\vee Party$ -> $Party$

$Party$, $\neg Party$ -> False

Por lo tanto, no es satisfactible.

Entonces $\{\neg Food \vee Party\vee\neg Drinks\vee Party\} \vdash \neg Food \vee \neg Drinks \vee Party$

\section*{7.4}

a) $\exists x (Parent(Joan,x)\wedge Female(x))$

b) $\exists^1 x (Parent(Joan,x)\wedge Female(x))$

c) $\exists^1 x Parent(Joan,x) \wedge \forall x (Parent(Joan,x) \rightarrow Female(x))$

d) $\exists^1 x (Parent(Joan,x) \wedge Parent(Kevin,x))$

e) $\exists x (Parent(Joan,x) \wedge Parent(Kevin,x))\wedge$ $\forall x (Parent(Joan,x) \rightarrow Parent(Kevin,x))$

\section*{8.10}

a. $Occupation(Emily,Surgeon) \vee Occupation(Emily, Lawyer)$

b. $Occupation(Joe,Actor)\wedge \exists x (\neg x = Actor \wedge Occupation(Joe,x)$

c. $\forall x (Occupation(x,Surgeon) \rightarrow Occupation(x,Doctor))$

d. $\forall x (Occupation(x,Lawyer) \rightarrow \neg Customer(Joe,x))$

e. $\exists x (Boss(x,Emily) \wedge Occupation(x,Lawyer))$

f. $\exists x (Occupation(x,Lawyer)\wedge \forall y (Customer(y,x)\rightarrow Occupation(y,Doctor))$

g. $\forall x (Occupation(x,Surgeon) \rightarrow \exists y (Occupation(y, Lawyer)\wedge Customer(x,y)))$

\section*{9.6}

a. $\forall x  (horse(x)\rightarrow mammal(x))$, $\forall x (Cow(x) \rightarrow mammal(Cows))$, $\forall x (pig(x)\rightarrow mammal(Pigs))$

b. $\forall x \forall y (horse(x)\wedge offspring(y,x) \rightarrow horse(y))$

c. $horse(Bluebeard)$

d. $parent(Bluebeard,Charlie)$

e. $\forall x \forall y(offspring(x,y) \rightarrow parent(y,x))$, $\forall x \forall y (parent(y,x) \rightarrow offspring(x,y))$

f. $\forall x \exists y (mammal(x) \rightarrow parent(y,x))$


\end{document}
